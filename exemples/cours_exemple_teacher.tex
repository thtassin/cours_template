\documentclass[12pt, fleqn]{book}
\usepackage[]{cours}
\usepackage{lipsum}

\begin{document}
\dominitoc
\tableofcontents
\chapter{Exemple de cours}
\begin{objectif}
	\item Présenter les différentes 
	fonctionnalités fournit par \texttt{cours.sty}
	\item Montrer un exemple de chaque environnement
\end{objectif}

\section{Présentation des environnements}
L'option \texttt{student} n'est pas activée lors de la compilation de ce document.
Elle permet de créer un texte à trou à l'aide de la commande
\texttt{blank}. Dans ce texte, il \blank{manque} un mot lorsque \texttt{student}
est activée.
\subsection{Définition}
\begin{defn}{Une définition}
	Cette environnement permet de mettre en valeur les défintions 
	apparaissant dans le cours. La couleur de l'arrière plan peut-être
	modifiée via \texttt{theme.sty}
\end{defn}
\subsection{Exemple}
Cette définition peut-être illustrée par un exemple dont le contenu est effacé
lorsque l'option \texttt{student} est précisée.
\begin{exemple}
	\lipsum[1]
\end{exemple}
\subsection{Attention}
Il est parfois important de mettre en avant certains subtilités.
\begin{attention}
	\lipsum[1]
\end{attention}
\subsection{Remarque}
Enfin quelques remarques sont toujours les bienvenues.
\begin{rema}
	Ceci est une remarque.
\end{rema}

\section{Une deuxième section}
\section{Une troisième section}

\begin{td}{On peut passer aux exercices}
	L'appel de l'environnement \texttt{td} ajoute une entrée à la table
	des matières. Chaque exercice est défini via le nouveau type de section 
	\texttt{exercice} prodiqué par le paquet.
\exercice{Ceci est un premier exercice}
\begin{exlist}
\item Question 1
	\begin{exlist}
	\item Sous-question 1
	\item Sous-question 2
	\end{exlist}
\item Question 2
\end{exlist}

\exercice{Un deuxième exercice}
\end{td}

\begin{corr}{On peut passer à la correction}
	L'appel de l'environnement \texttt{corr} ajoute une entrée à la table
	des matières. Chaque correction est défini via le nouveau type de section 
	\texttt{correction} prodiqué par le paquet. La correction 
	disparaît lorsque l'option \texttt{student} est activée.
\correction{Ceci est un premier exercice}
\begin{corrlist}
\item Question 1
	\begin{corrlist}
	\item Sous-question 1
	\item Sous-question 2
	\end{corrlist}
\item Question 2
\end{corrlist}
\correction{Un deuxième exercice}
\end{corr}
\end{document}
