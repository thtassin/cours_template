\documentclass{book}
\usepackage[]{cours}
\usepackage{lipsum}

\begin{document}

\chapter{Exemple de cours}

\section{Présentation des environnements}
L'option \texttt{student} n'est pas activée lors de la compilation de ce document.
Elle permet de créer un texte à trou à l'aide de la commande
\texttt{blank}. Dans ce texte, il \blank{manque} un mot lorsque \texttt{student}
est activée.
\subsection{Définition}
\begin{defn}
	Cette environnement permet de mettre en valeur les défintions 
	apparaissant dans le cours. La couleur de l'arrière plan peut-être
	modifiée via \texttt{theme.sty}
\end{defn}

\subsection{Attention}
Il est parfois important de mettre en avant certains subtilités.
\begin{attention}
	\lipsum[1]
\end{attention}

\subsection{Application}
Il est parfois utile de faire une application.
\begin{application}
	\lipsum[1]
\end{application}

\subsection{Remarque}
Enfin quelques remarques sont toujours les bienvenues.
\begin{rema}
	Ceci est une remarque.
\end{rema}

\subsection{Au tableau}
\begin{tableau}
 \lipsum[2].
\end{tableau}

\subsection{Équation}
\begin{eqn}{Équation de Navier-Stokes}
 \lipsum[2].
\end{eqn}

\section{Une deuxième section}
\section{Une troisième section}

\chapter{Un autre chapitre}

\section{Avec une autre section}
\end{document}
